\documentclass[a4paper,12pt]{article}
\usepackage{amsmath,amssymb}
\usepackage{graphicx}
\usepackage{scaladefs}
\usepackage{scalit}
\usepackage{fancyhdr}
\pagestyle{fancy}
\lhead{\today}
\rhead{util/tools.nw}
\begin{document}
\section{Command-line tools}
The literate programming tools can always be accessed by directly calling the
corresponding main class, for example:

\begin{verbatim}
scala scalit.tangle.LitComp inputfile
\end{verbatim}

However, this is quite verbose on the long run. To facilitate, the shell scripts
\texttt{sweave} and \texttt{litcomp} provide an easy command line access.

\subsection{Wrapping the weave tool}
To wrap the weave tool, we simply call scala with the corresponding class:

$\left<\mbox{\emph{sweave}}\right>\equiv$
\begin{verbatim}#!/bin/sh
#
# Scalit - Call weave with the arguments
#
scala scalit.weave.Weave $@

\end{verbatim}
\subsection{Wrapping the Compiler utility}
Also the compiler utility will be used a lot. It is wrapped in the same way
that weave is.

$\left<\mbox{\emph{litcomp}}\right>\equiv$
\begin{verbatim}#!/bin/sh
#
# Scalit - Call tangle with arguments
#
scala scalit.tangle.LitComp $@
\end{verbatim}
\end{document}


