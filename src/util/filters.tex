\documentclass[a4paper,12pt]{article}
\usepackage{amsmath,amssymb}
\usepackage{graphicx}
\usepackage{scaladefs}
\usepackage{scalit}
\usepackage{fancyhdr}
\pagestyle{fancy}
\lhead{\today}
\rhead{util/filters.nw}
\begin{document}
\section{A filter mechanism for Scalit}
The literate programming mechanism proposed until now is a very static thing,
you only have a few choices (whether you want syntax highlighting, an index
etc.). Things that you possibly would want to do is to convert a file from
its LaTeX formatting into HTML, or add other syntax highlighting.

This file demonstrates a simple filter mechanism for Scalit, which is
based on intervention in two stages: On the markup level and on the
block level. Such filter modules can then be loaded by using reflection.

$\left<\mbox{\emph{*}}\right>\equiv$
\begin{program}{\vem package}~scalit.util
\\[0.5em]$<$Filtering~on~Markup~level$>$
\\[0.5em]$<$Sample~markup~filters$>$
\\[0.5em]$<$Filtering~on~Block~level$>$
\\[0.5em]$<$Sample~block~filters$>$
\\[0.5em]\end{program}


\subsection{Filtering on the Markup level}
Applying a filter on the markup level is quite limited in its capabilities
as we only have access to information encoded in these lines. Nevertheless,
for example a LaTeX-to-HTML converter could be implemented on this level.

These filters are basically functions from the input stream of lines to
an altered stream of lines, therefore we inherit from this function:

$\left<\mbox{\emph{Filtering on Markup level}}\right>\equiv$
\begin{program}{\vem import}~markup.Line
\\{\vem abstract}~{\vem class}~MarkupFilter~{\vem extends}
\\~~~~$($Stream$[$Line$]$~$\Rightarrow$~Stream$[$Line$]$$)$~{\small\{}
\\~~~~$<$Feed~external~utility~{\vem with}~lines$>$
\\{\small\}}
\\[0.5em]\end{program}\classdefinition{MarkupFilter}



Note that such filters can also wrap standard Unix tools by feeding them
the lines on standard input and parsing the filtered output with a MarkupParser.

$\left<\mbox{\emph{Feed external utility with lines}}\right>\equiv$
\begin{program}~~~~{\vem def}~externalFilter$($command\,{\rm :}~String,
\\~~~~~~~~~~~~~~~~~~~~~~~~~~~~~~~~~~~~~~~~~~lines\,{\rm :}~Stream$[$Line$]$$)${\rm :}~Stream$[$Line$]$~=~{\small\{}
\end{program}\methoddefinition{MarkupFilter}{externalFilter}



We take it we are dealing with utilities that take the markup format stream
as standard input. If that is not the case, then you will have to write a custom
function.

The following code is very Java-intensive and will therefore have to be changed
as soon as Scala gets its own form of process control.

First, we'll have to start the process:

$\left<\mbox{\emph{Feed external utility with lines}}\right>+\equiv$
\begin{program}~~~~~~~~{\vem val}~p~=~Runtime.getRuntime$($$)$.exec$($command$)$
\\[0.5em]\end{program}


Ok, now we'll have to pass this process a print-out of the lines. We'll do this
using a print writer:

$\left<\mbox{\emph{Feed external utility with lines}}\right>+\equiv$
\begin{program}~~~~~~~~{\vem val}~procWriter~=~{\vem new}~java.io.PrintWriter$($p.getOutputStream$($$)$$)$
\\[0.5em]\end{program}


Now let us write to this process:

$\left<\mbox{\emph{Feed external utility with lines}}\right>+\equiv$
\begin{program}~~~~~~~~lines~foreach~procWriter.println
\\[0.5em]\end{program}


The input will now be obtained from the input stream, parsed again in line format.
For this, we use the conversion utility described in \texttt{uti/conversions.nw},
\texttt{linesFromLiterateInput}. We then return this result

$\left<\mbox{\emph{Feed external utility with lines}}\right>+\equiv$
\begin{program}~~~~~~~~procWriter.close$($$)$
\\~~~~~~~~p.waitFor$($$)$
\\~~~~~~~~util.conversions.linesFromMarkupInput$($p.getInputStream$($$)$$)$
\\~~~~{\small\}}
\\[0.5em]\end{program}


\subsubsection{Tee: A sample line filter}
To exemplify how such an external tool could be used, the following filter calls
the unix utility \texttt{tee}, which just replicates its input on the output while
also writing to a file.

$\left<\mbox{\emph{Sample markup filters}}\right>\equiv$
\begin{program}{\vem class}~tee~{\vem extends}~MarkupFilter~{\small\{}
\\~~~~{\vem def}~apply$($lines\,{\rm :}~Stream$[$Line$]$$)${\rm :}~Stream$[$Line$]$~=~{\small\{}
\\~~~~~~~~externalFilter$($"tee~out",lines$)$
\\~~~~{\small\}}
\\{\small\}}
\\[0.5em]\end{program}\classdefinition{tee}
\methoddefinition{tee}{apply}



\subsubsection{SimpleSubst: Another example}
Filters can also be written in pure Scala, avoiding the overhead of calling an
external process and reparsing the markup lines. The following example replaces
the sequence "LaTeX" with its nice form: \LaTeX

$\left<\mbox{\emph{Sample markup filters}}\right>+\equiv$
\begin{program}{\vem class}~simplesubst~{\vem extends}~MarkupFilter~{\small\{}
\\~~~~{\vem def}~apply$($lines\,{\rm :}~Stream$[$Line$]$$)${\rm :}~Stream$[$Line$]$~=~{\small\{}
\\~~~~~~~~{\vem import}~markup.TextLine
\\~~~~~~~~lines~map~{\small\{}
\\~~~~~~~~~~~~{\vem case}~TextLine$($cont$)$~$\Rightarrow$
\\~~~~~~~~~~~~~~~~TextLine$($cont.replace$($"~"~$+$~"LaTeX~","~$\backslash$$\backslash$LaTeX~"$)$$)$
\\~~~~~~~~~~~~{\vem case}~unchanged~$\Rightarrow$~unchanged
\\~~~~~~~~{\small\}}
\\~~~~{\small\}}
\\{\small\}}
\\[0.5em]\end{program}\classdefinition{simplesubst}
\methoddefinition{simplesubst}{apply}



Now that the filter is defined, using it is as simple as invoking

\begin{verbatim}
sweave -lfilter util.simplesubst lp.nw -o lp.tex
\end{verbatim}

\subsection{Filtering on the Block level}
Filters on the chunk level are already far more powerful. We already
have a high-level view of the document and can do the same level of analysis
that we use for the index generation (which is not in filter form).

What we get here as input is the stream of blocks and we return an altered
stream of blocks. At the moment, no further help is given to the programmer
in the form of utility functions, so we basically can define the interface
in one line:

$\left<\mbox{\emph{Filtering on Block level}}\right>\equiv$
\begin{program}{\vem import}~markup.Block
\\{\vem abstract}~{\vem class}~BlockFilter~{\vem extends}
\\~~~~$($Stream$[$Block$]$~$\Rightarrow$~Stream$[$Block$]$$)$~{\small\{}
\\{\small\}}
\\[0.5em]\end{program}\classdefinition{BlockFilter}



\subsubsection{A block-level example: Stats}
While we do not have much help for filters, writing them still is not
too hard, as this simple example (that collects statistics on how
many lines of code vs how many lines of documentation were provided).

We could also access the tangled output, but in the interest of simplicity,
this example will only deal with the unstructured blocks:

$\left<\mbox{\emph{Sample block filters}}\right>\equiv$
\begin{program}{\vem class}~stats~{\vem extends}~BlockFilter~{\small\{}
\\~~~~{\vem def}~apply$($blocks\,{\rm :}~Stream$[$Block$]$$)${\rm :}~Stream$[$Block$]$~=~{\small\{}
\\~~~~~~~~{\vem val}~$($doclines,codelines$)$~=~collectStats$($blocks$)$
\\[0.5em]\end{program}\classdefinition{stats}
\methoddefinition{stats}{apply}



With these values, we can build a new documentation block
holding them:

$\left<\mbox{\emph{Sample block filters}}\right>+\equiv$
\begin{program}~~~~~~~~{\vem import}~markup.{\small\{}TextLine,NewLine{\small\}}
\\~~~~~~~~{\vem val}~content~=
\\~~~~~~~~~~~~Stream.cons$($NewLine,
\\~~~~~~~~~~~~Stream.cons$($TextLine$($"Documentation~lines\,{\rm :}~"~$+$
\\~~~~~~~~~~~~~~~~~~~~~~~~~~~~~~~~~~~~doclines~$+$
\\~~~~~~~~~~~~~~~~~~~~~~~~~~~~~~~~~~~~",~Code~lines\,{\rm :}~"~$+$
\\~~~~~~~~~~~~~~~~~~~~~~~~~~~~~~~~~~~~codelines$)$,
\\~~~~~~~~~~~~Stream.cons$($NewLine,Stream.empty$)$$)$$)$
\\~~~~~~~~Stream.concat$($blocks,Stream.cons$($
\\~~~~~~~~~~~~markup.DocuBlock$($$-$1,$-$1,content$)$,Stream.empty$)$$)$
\\~~~~{\small\}}
\\[0.5em]\end{program}


Now for the main collection function: It just traverses the
content of the blocks in unprocessed form:

$\left<\mbox{\emph{Sample block filters}}\right>+\equiv$
\begin{program}~~~~{\vem def}~collectStats$($bs\,{\rm :}~Stream$[$Block$]$$)${\rm :}~$($Int,Int$)$~=~{\small\{}
\\~~~~~~~~{\vem def}~collectStats0$($str\,{\rm :}~Stream$[$Block$]$,
\\~~~~~~~~~~~~~~~~~~~~~~~~~~~~~~~~doclines\,{\rm :}~Int,
\\~~~~~~~~~~~~~~~~~~~~~~~~~~~~~~~~codelines\,{\rm :}~Int$)${\rm :}~$($Int,Int$)$~=~str~{\vem match}~{\small\{}
\\~~~~~~~~~~~~~~{\vem case}~Stream.empty~$\Rightarrow$~$($doclines,codelines$)$
\\~~~~~~~~~~~~~~{\vem case}~Stream.cons$($first,rest$)$~$\Rightarrow$~first~{\vem match}~{\small\{}
\\[0.5em]\end{program}\methoddefinition{stats}{collectStats}



If a code block is encountered, then we will just increment the
number of lines of code:

$\left<\mbox{\emph{Sample block filters}}\right>+\equiv$
\begin{program}~~~~~~~~~~~~~~~~~~{\vem case}~markup.CodeBlock$($\_,\_,lines,\_$)$~$\Rightarrow$
\\~~~~~~~~~~~~~~~~~~~~~~{\vem val}~codels~=~$($lines~foldLeft~0$)$~{\small\{}
\\~~~~~~~~~~~~~~~~~~~~~~~~~~$($acc\,{\rm :}~Int,~l\,{\rm :}~markup.Line$)$~$\Rightarrow$~l~{\vem match}~{\small\{}
\\~~~~~~~~~~~~~~~~~~~~~~~~~~~~~~{\vem case}~markup.NewLine~$\Rightarrow$~acc~$+$~1
\\~~~~~~~~~~~~~~~~~~~~~~~~~~~~~~{\vem case}~\_~$\Rightarrow$~acc
\\~~~~~~~~~~~~~~~~~~~~~~~~~~{\small\}}
\\~~~~~~~~~~~~~~~~~~~~~~{\small\}}
\\~~~~~~~~~~~~~~~~~~~~~~collectStats0$($rest,doclines,codelines~$+$~codels$)$
\\[0.5em]\end{program}


The documentation block code is similiar, again with an accumulator for
the lines:

$\left<\mbox{\emph{Sample block filters}}\right>+\equiv$
\begin{program}~~~~~~~~~~~~~~~~~~{\vem case}~markup.DocuBlock$($\_,\_,lines$)$~$\Rightarrow$
\\~~~~~~~~~~~~~~~~~~~~~~{\vem val}~doculs~=~$($lines~foldLeft~0$)$~{\small\{}
\\~~~~~~~~~~~~~~~~~~~~~~~~~~$($acc\,{\rm :}~Int,~l\,{\rm :}~markup.Line$)$~$\Rightarrow$~l~{\vem match}~{\small\{}
\\~~~~~~~~~~~~~~~~~~~~~~~~~~~~~~{\vem case}~markup.NewLine~$\Rightarrow$~acc~$+$~1
\\~~~~~~~~~~~~~~~~~~~~~~~~~~~~~~{\vem case}~\_~$\Rightarrow$~acc
\\~~~~~~~~~~~~~~~~~~~~~~~~~~{\small\}}
\\~~~~~~~~~~~~~~~~~~~~~~{\small\}}
\\~~~~~~~~~~~~~~~~~~~~~~collectStats0$($rest,doclines~$+$~doculs,~codelines$)$
\\~~~~~~~~~~~~~~{\small\}}
\\~~~~~~~~{\small\}}
\\~~~~~~~~collectStats0$($bs,0,0$)$
\\~~~~{\small\}}
\\{\small\}}
\\[0.5em]\end{program}


Invoking this filter is as easy as calling

\begin{verbatim}
sweave -bfilter util.stats somefile.nw -o somefile.tex
\end{verbatim}
\end{document}


