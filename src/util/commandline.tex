\documentclass[a4paper,12pt]{article}
\usepackage{amsmath,amssymb}
\usepackage{graphicx}
\usepackage{scaladefs}
\usepackage{scalit}
\usepackage{fancyhdr}
\pagestyle{fancy}
\lhead{\today}
\rhead{util/commandline.nw}
\begin{document}
\section{Support for command line arguments}
All the different command line tools produced (\texttt{tangle}, \texttt{weave},
the one-step-compiler) have some things in common: They take as input
either markup files or literate files. Also, it would be quite useful to
specify more than one of them, for example to create a woven document out of
several input files. The class \texttt{LiterateSettings} will provide exactly this
functionality. It is not as evolved as the compiler settings and will just
remember arguments that it does not know about.

$\left<\mbox{\emph{*}}\right>\equiv$
\begin{program}{\vem package}~scalit.util
\\{\vem import}~scalit.markup.\_
\\[0.5em]{\vem object}~LiterateSettings~{\small\{}
\\~~~~$<$getting~the~arguments$>$
\\{\small\}}
\\[0.5em]{\vem class}~LiterateSettings$(${\vem val}~settings\,{\rm :}~Map$[$String,List$[$String$]$$]$,
\\~~~~~~~~~~~~~~~~~~~~~~~~~~~~~~~~~~~~~~~~~~~~~~ls\,{\rm :}~List$[$Stream$[$Line$]$$]$$)$~{\small\{}
\\~~~~$<$Constructor~{\vem with}~argument~list$>$
\\[0.5em]~~~~$<$a~reference~to~output$>$
\\~~~~$<$getting~the~filters$>$
\\~~~~$<$getting~the~lines$>$
\\~~~~$<$getting~the~blocks$>$
\\~~~~$<$getting~the~chunks$>$
\\{\small\}}
\\[0.5em]\end{program}\classdefinition{LiterateSettings}
\objectdefinition{LiterateSettings}
\valuedefinition{LiterateSettings}{settings }
\valuedefinition{LiterateSettings}{ls}



We have two fields here: One is for all the settings that we got and
the other is for the input files. But we won't really use this constructor:
We'd rather directly take the arguments given to the application.

\subsection{Parsing the command line arguments}
The usual way to call \texttt{LiterateSettings} is with an argument list in
form of an array of strings. Inside this constructor, we'll obtain the
value for settings and lines with a call to a recursive function.

$\left<\mbox{\emph{Constructor with argument list}}\right>\equiv$
\begin{program}~~~~{\vem def}~{\vem this}$($p\,{\rm :}~$($Map$[$String,List$[$String$]$$]$,List$[$Stream$[$Line$]$$]$$)$$)$~=
\\~~~~~~~~{\vem this}$($p.\_1,~p.\_2$)$
\\[0.5em]~~~~{\vem def}~{\vem this}$($args\,{\rm :}~Array$[$String$]$$)$~=
\\~~~~~~~~{\vem this}$($LiterateSettings.getArgs$($args.toList,Map$($$)$,Nil$)$$)$
\\[0.5em]\end{program}\methoddefinition{LiterateSettings}{$\langle$init$\rangle$}
\methoddefinition{LiterateSettings}{$\langle$init$\rangle$}



Now to get the arguments, we go element for element through
the list, trying to obtain something: This function has to be defined
outside of the class \texttt{LiterateSettings}, because it will be called
before the object exists.

$\left<\mbox{\emph{getting the arguments}}\right>\equiv$
\begin{program}~~~~{\vem def}~getArgs$($args\,{\rm :}~List$[$String$]$,~settings\,{\rm :}~Map$[$String,List$[$String$]$$]$,
\\~~~~~~~~~~~~~~~~~~~~~~~~~~~~lines\,{\rm :}~List$[$Stream$[$Line$]$$]$$)$~{\rm :}
\\~~~~~~~~~~~~$($Map$[$String,List$[$String$]$$]$,List$[$Stream$[$Line$]$$]$$)$~=~args~{\vem match}~{\small\{}
\\[0.5em]\end{program}\methoddefinition{LiterateSettings}{getArgs}



If any filename is prefixed by an argument consisting of \texttt{-m},
then we parse some lines from markup input:

$\left<\mbox{\emph{getting the arguments}}\right>+\equiv$
\begin{program}{\vem case}~"$-$m"~{\rm :}{\rm :}~markupfile~{\rm :}{\rm :}~xs~$\Rightarrow$~{\small\{}
\\~~~~{\vem val}~mlines~=~conversions.linesFromMarkupFile$($markupfile$)$
\\~~~~getArgs$($xs,settings,mlines~{\rm :}{\rm :}~lines$)$
\\{\small\}}
\\[0.5em]\end{program}


We might also read from standard input. With the command line option
of \texttt{-li}, we try to read a literate program, with \texttt{-mi}, we try
to read markup input:

$\left<\mbox{\emph{getting the arguments}}\right>+\equiv$
\begin{program}{\vem case}~"$-$li"~{\rm :}{\rm :}~Nil~$\Rightarrow$~{\small\{}
\\~~~~{\vem val}~llines~=~conversions.linesFromLiterateInput$($System.in$)$
\\~~~~$($settings,~lines.reverse~{\rm :}{\rm :}{\rm :}~List$($llines$)$$)$
\\{\small\}}
\\{\vem case}~"$-$mi"~{\rm :}{\rm :}~Nil~$\Rightarrow$~{\small\{}
\\~~~~{\vem val}~mlines~=~conversions.linesFromMarkupInput$($System.in$)$
\\~~~~$($settings,lines.reverse~{\rm :}{\rm :}{\rm :}~List$($mlines$)$$)$
\\{\small\}}
\\[0.5em]\end{program}


\subsubsection{Additional command line arguments}
If the frontmost element of \texttt{args} begins with \texttt{-}, that means we
are dealing with an option that takes one argument (all the other cases
were dealt with before). We append it to the list of arguments already
given with this option

$\left<\mbox{\emph{getting the arguments}}\right>+\equiv$
\begin{program}{\vem case}~opt~{\rm :}{\rm :}~arg~{\rm :}{\rm :}~xs~$\Rightarrow$
\\~~~~{\vem if}$($~opt$($0$)$~$==$~'$-$'~$)$
\\~~~~~~~~getArgs$($xs,settings~$+$
\\~~~~~~~~~~~~~~~~~~~~~~~~$($opt~$\rightarrow$~$($arg~{\rm :}{\rm :}~settings.getOrElse$($opt,Nil$)$$)$$)$,
\\~~~~~~~~~~~~~~~~~~~~~~lines$)$
\\~~~~{\vem else}~{\small\{}
\\~~~~~~~~{\vem val}~llines~=~conversions.linesFromLiterateFile$($opt$)$
\\~~~~~~~~getArgs$($arg~{\rm :}{\rm :}~xs,~settings,~llines~{\rm :}{\rm :}~lines$)$
\\~~~~{\small\}}
\\[0.5em]\end{program}


We have to treat the case where we get \texttt{``-o''} specially:
in this case, we'll have to provide output to a file:

$\left<\mbox{\emph{a reference to output}}\right>\equiv$
\begin{program}~~{\vem lazy}~{\vem val}~output\,{\rm :}~java.io.PrintStream~=
\\~~~~~~settings~get~"$-$o"~{\vem match}~{\small\{}
\\~~~~~~~~~~{\vem case}~None~$\Rightarrow$~System.out
\\~~~~~~~~~~{\vem case}~Some$($List$($file$)$$)$~$\Rightarrow$~{\vem new}~java.io.PrintStream$($
\\~~~~~~~~~~~~~~{\vem new}~java.io.FileOutputStream$($file$)$
\\~~~~~~~~~~$)$
\\~~~~~~{\small\}}
\\[0.5em]\end{program}


Finally, if there is only one argument left, then it has to be
an input from a literate file, otherwise it would have been treated:

$\left<\mbox{\emph{getting the arguments}}\right>+\equiv$
\begin{program}{\vem case}~litfile~{\rm :}{\rm :}~xs~$\Rightarrow$~{\small\{}
\\~~~~{\vem val}~llines~=~conversions.linesFromLiterateFile$($litfile$)$
\\~~~~getArgs$($xs,settings,llines~{\rm :}{\rm :}~lines$)$
\\{\small\}}
\\{\vem case}~Nil~$\Rightarrow$~$($settings,lines.reverse$)$
\\{\small\}}
\\[0.5em]\end{program}


\subsection{Filters}
We can also, on the command line specify filters to be applied to the
markup or block phase. Filters are just classes extending from
\texttt{MarkupFilter} or \texttt{BlockFilter}. The following two fields hold
reference to these filters:

$\left<\mbox{\emph{getting the filters}}\right>\equiv$
\begin{program}~~~~{\vem import}~scalit.util.{\small\{}MarkupFilter,BlockFilter{\small\}}
\\~~~~{\vem val}~markupFilters\,{\rm :}~List$[$MarkupFilter$]$~=
\\~~~~~~~~settings~get~$($"$-$lfilter"$)$~{\vem match}~{\small\{}
\\~~~~~~~~~~~~{\vem case}~None~$\Rightarrow$~Nil
\\~~~~~~~~~~~~{\vem case}~Some$($xs$)$~$\Rightarrow$~{\small\{}
\\[0.5em]\end{program}


If we have some names, then we will try to load them using reflection,
creating an instance for each class.

$\left<\mbox{\emph{getting the filters}}\right>+\equiv$
\begin{program}xs~map~{\small\{}
\\~~~~name~$\Rightarrow$
\\~~~~~~~~{\vem try}~{\small\{}
\\~~~~~~~~~~~~{\vem val}~filterClass~=~Class.forName$($name$)$
\\~~~~~~~~~~~~filterClass.newInstance.asInstanceOf$[$MarkupFilter$]$
\\~~~~~~~~{\small\}}~{\vem catch}~{\small\{}
\\~~~~~~~~~~~~{\vem case}~ex~$\Rightarrow$
\\~~~~~~~~~~~~~~~~Console.err.println$($"Could~not~load~filter~"~$+$~name$)$
\\~~~~~~~~~~~~~~~~System.exit$($1$)$
\end{program}


Somehow, System.exit is not enough for the type checker, therefore we will
have to give back some dummy class if that happens.

$\left<\mbox{\emph{getting the filters}}\right>+\equiv$
\begin{program}~~~~~~~~~~~~~~~~{\vem new}~util.tee
\\~~~~~~~~{\small\}}
\\~~~~~~{\small\}}
\\~~~~{\small\}}
\\{\small\}}
\\[0.5em]\end{program}


With block filters, it's exactly the same story:

$\left<\mbox{\emph{getting the filters}}\right>+\equiv$
\begin{program}{\vem lazy}~{\vem val}~blockFilters\,{\rm :}~List$[$BlockFilter$]$~=
\\~~~~settings~get~"$-$bfilter"~{\vem match}~{\small\{}
\\~~~~~~~~{\vem case}~None~$\Rightarrow$~Nil
\\~~~~~~~~{\vem case}~Some$($xs$)$~$\Rightarrow$~xs~map~{\small\{}
\\~~~~~~~~~~~~name~$\Rightarrow$
\\~~~~~~~~~~~~~~~~{\vem try}~{\small\{}
\\~~~~~~~~~~~~~~~~~~~~{\vem val}~filterClass~=~Class.forName$($name$)$
\\~~~~~~~~~~~~~~~~~~~~filterClass.newInstance.asInstanceOf$[$BlockFilter$]$
\\~~~~~~~~~~~~~~~~{\small\}}~{\vem catch}~{\small\{}
\\~~~~~~~~~~~~~~~~~~~~{\vem case}~e~$\Rightarrow$
\\~~~~~~~~~~~~~~~~~~~~~~~~Console.err.println$($"Could~not~load"~$+$
\\~~~~~~~~~~~~~~~~~~~~~~~~"~block~filter~"~$+$~name$)$
\\~~~~~~~~~~~~~~~~~~~~~~~~System.exit$($1$)$
\\~~~~~~~~~~~~~~~~~~~~~~~~{\vem new}~util.stats
\\~~~~~~~~~~~~~~~~{\small\}}
\\~~~~~~~~{\small\}}
\\~~~~{\small\}}
\\[0.5em]\end{program}


\subsection{Getting the content of the lines in different formats}
With the settings in place, it is very easy to get the actual content
of the files in different formats. We will just have to take care of the
filters:

$\left<\mbox{\emph{getting the lines}}\right>\equiv$
\begin{program}~~~~{\vem lazy}~{\vem val}~lines\,{\rm :}~List$[$Stream$[$Line$]$$]$~=~ls~map~{\small\{}
\\~~~~~~~~markupStream\,{\rm :}~Stream$[$Line$]$~$\Rightarrow$~markupStream
\\~~~~~~~~~~~~$($markupFilters~foldLeft~markupStream$)$~{\small\{}
\\~~~~~~~~~~~~~~~~$($acc\,{\rm :}~Stream$[$Line$]$,~f\,{\rm :}~MarkupFilter$)$~$\Rightarrow$~f$($acc$)$
\\~~~~~~~~~~~~{\small\}}
\\~~~~{\small\}}
\\[0.5em]\end{program}


To build the blocks, we'll just have to instantiate a block builder
for every stream of lines. Also, we'll have to apply the filters in order,
of course.

$\left<\mbox{\emph{getting the blocks}}\right>\equiv$
\begin{program}~~~~{\vem val}~blocks\,{\rm :}~List$[$$($Stream$[$markup.Block$]$,String$)$$]$~=~lines~map~{\small\{}
\\~~~~~~~~l~$\Rightarrow$~{\small\{}
\\~~~~~~~~~~~~{\vem val}~bb~=~BlockBuilder$($l$)$
\\~~~~~~~~~~~~{\vem val}~filteredBlocks\,{\rm :}~Stream$[$Block$]$~=
\\~~~~~~~~~~~~~~~~$($blockFilters~foldLeft~bb.blocks$)$~{\small\{}
\\~~~~~~~~~~~~~~~~~~~~$($acc\,{\rm :}~Stream$[$Block$]$,f\,{\rm :}~BlockFilter$)$~$\Rightarrow$~f$($acc$)$
\\~~~~~~~~~~~~~~~~{\small\}}
\\~~~~~~~~~~~~$($filteredBlocks,bb.filename$)$
\\~~~~~~~~{\small\}}
\\~~~~{\small\}}
\\[0.5em]\end{program}


The chunk collections work in a very similar way.

$\left<\mbox{\emph{getting the chunks}}\right>\equiv$
\begin{program}{\vem import}~scalit.tangle.emptyChunkCollection
\\{\vem lazy}~{\vem val}~chunkCollections~=~blocks~map~{\small\{}
\\~~~~{\vem case}~$($bs,name$)$~$\Rightarrow$
\\~~~~emptyChunkCollection$($name$)$~addBlocks~conversions.codeblocks$($bs$)$
\\{\small\}}
\end{program}


\end{document}


