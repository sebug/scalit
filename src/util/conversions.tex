\documentclass[a4paper,12pt]{article}
\usepackage{amsmath,amssymb}
\usepackage{graphicx}
\usepackage{scaladefs}
\usepackage{scalit}
\usepackage{fancyhdr}
\pagestyle{fancy}
\lhead{\today}
\rhead{util/conversions.nw}
\begin{document}
\section{Conversions}
With all the literate programming utilities in place, we will want to access
different stages without always setting up a \texttt{MarkupGenerator}, reading
files etc. The following conversion functions will prove useful:

$\left<\mbox{\emph{*}}\right>\equiv$
\begin{program}{\vem package}~scalit.util
\\[0.5em]{\vem object}~conversions~{\small\{}
\\~~~~$<$to~line~format$>$
\\[0.5em]~~~~$<$to~block~format$>$
\\{\small\}}
\\[0.5em]\end{program}\objectdefinition{conversions}



\subsection{Conversions to line format}
The line format will usally be the first step. It is usually either generated
from a file or from standard input:

$\left<\mbox{\emph{to line format}}\right>\equiv$
\begin{program}{\vem import}~java.io.{\small\{}BufferedReader,FileReader,InputStreamReader{\small\}}
\\{\vem import}~scala.util.parsing.input.StreamReader
\\[0.5em]{\vem import}~markup.{\small\{}Line,MarkupGenerator{\small\}}
\\[0.5em]{\vem def}~linesFromLiterateFile$($filename\,{\rm :}~String$)${\rm :}~Stream$[$Line$]$~=~{\small\{}
\\~~~~{\vem val}~input~=~StreamReader$($
\\~~~~~~~~~~~~~~~~~~~~~~~~~~~~~~~~{\vem new}~BufferedReader$($
\\~~~~~~~~~~~~~~~~~~~~~~~~~~~~~~~~{\vem new}~FileReader$($filename$)$$)$$)$
\\~~~~$(${\vem new}~MarkupGenerator$($input,filename$)$$)$.lines
\\{\small\}}
\\[0.5em]{\vem def}~linesFromLiterateInput$($in\,{\rm :}~java.io.InputStream$)${\rm :}~Stream$[$Line$]$~=~{\small\{}
\\~~~~{\vem val}~input~=~StreamReader$(${\vem new}~InputStreamReader$($in$)$$)$
\\~~~~$(${\vem new}~MarkupGenerator$($input,""$)$$)$.lines
\\{\small\}}
\\[0.5em]\end{program}\methoddefinition{conversions}{linesFromLiterateFile}
\methoddefinition{conversions}{linesFromLiterateInput}



We could, of course also get input in markup format. This is treated in
the class \texttt{MarkupReader}:

$\left<\mbox{\emph{to line format}}\right>+\equiv$
\begin{program}{\vem import}~markup.MarkupParser
\\{\vem def}~linesFromMarkupFile$($filename\,{\rm :}~String$)${\rm :}~Stream$[$Line$]$~=~{\small\{}
\\~~~~{\vem val}~input~=~StreamReader$($
\\~~~~~~~~~~~~~~~~~~~~~~~~~~~~~~~~{\vem new}~BufferedReader$($
\\~~~~~~~~~~~~~~~~~~~~~~~~~~~~~~~~{\vem new}~FileReader$($filename$)$$)$$)$
\\[0.5em]~~~~$(${\vem new}~MarkupParser$($input$)$$)$.lines
\\{\small\}}
\\[0.5em]{\vem def}~linesFromMarkupInput$($in\,{\rm :}~java.io.InputStream$)${\rm :}~Stream$[$Line$]$~=~{\small\{}
\\~~~~{\vem val}~input~=~StreamReader$(${\vem new}~InputStreamReader$($in$)$$)$
\\~~~~$(${\vem new}~MarkupParser$($input$)$$)$.lines
\\{\small\}}
\\[0.5em]
\end{program}\methoddefinition{conversions}{linesFromMarkupFile}
\methoddefinition{conversions}{linesFromMarkupInput}



\subsection{Conversions to block format}
The block format takes a stream of lines as input, so we will have four similiar functions
that just call the corresponding line generating functions.

$\left<\mbox{\emph{to block format}}\right>\equiv$
\begin{program}{\vem import}~markup.{\small\{}BlockBuilder,Block{\small\}}
\\{\vem def}~blocksFromLiterateFile$($filename\,{\rm :}~String$)${\rm :}~Stream$[$Block$]$~=
\\~~~~BlockBuilder$($linesFromLiterateFile$($filename$)$$)$.blocks
\\[0.5em]{\vem def}~blocksFromLiterateInput$($in\,{\rm :}~java.io.InputStream$)${\rm :}~Stream$[$Block$]$~=
\\~~~~BlockBuilder$($linesFromLiterateInput$($in$)$$)$.blocks
\\[0.5em]{\vem def}~blocksFromMarkupFile$($filename\,{\rm :}~String$)${\rm :}~Stream$[$Block$]$~=
\\~~~~BlockBuilder$($linesFromMarkupFile$($filename$)$$)$.blocks
\\[0.5em]{\vem def}~blocksFromMarkupInput$($in\,{\rm :}~java.io.InputStream$)${\rm :}~Stream$[$Block$]$~=
\\~~~~BlockBuilder$($linesFromMarkupInput$($in$)$$)$.blocks
\\[0.5em]\end{program}\methoddefinition{conversions}{blocksFromLiterateFile}
\methoddefinition{conversions}{blocksFromLiterateInput}
\methoddefinition{conversions}{blocksFromMarkupFile}
\methoddefinition{conversions}{blocksFromMarkupInput}



Another demand will be to just get the code blocks (for tangle, for example).
We'll also have to make a (safe) downcast, unfortunately.

$\left<\mbox{\emph{to block format}}\right>+\equiv$
\begin{program}{\vem import}~markup.{\small\{}CodeBlock,DocuBlock{\small\}}
\\{\vem def}~codeblocks$($blocks\,{\rm :}~Stream$[$Block$]$$)${\rm :}~Stream$[$CodeBlock$]$~=
\\~~~~$($blocks~filter~{\small\{}
\\~~~~~~~~{\vem case}~c\,{\rm :}~CodeBlock~$\Rightarrow$~{\vem true}
\\~~~~~~~~{\vem case}~d\,{\rm :}~DocuBlock~$\Rightarrow$~{\vem false}
\\~~~~{\small\}}$)$.asInstanceOf$[$Stream$[$CodeBlock$]$$]$
\end{program}\methoddefinition{conversions}{codeblocks}



\end{document}


