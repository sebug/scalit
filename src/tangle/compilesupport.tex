\documentclass[a4paper,12pt]{article}
\usepackage{amsmath,amssymb}
\usepackage{graphicx}
\usepackage{scaladefs}
\usepackage{scalit}
\usepackage{fancyhdr}
\pagestyle{fancy}
\lhead{\today}
\rhead{tangle/compilesupport.nw}
\begin{document}
\section{A closer integration in the Scala compiler}
In the file \texttt{tangle.nw}, we outlined on how one could puzzle together a
compileable file out of a stream of blocks. This is very useful on its own,
but a closer compiler integration is desirable: This way, we can tell the
compiler the real line numbers of our literate program and not of the tangled
source - we will be able to debug more efficiently. Also, we can access
information provided by the compiler: What was defined by this block and
what variables will be used.

Concretely, we will be implementing \texttt{CoTangle} which will call the
compiler with the code taken from the stream of blocks. \texttt{LitComp} is
the command line application that directly compiles an input literate
program.

Additionally, we provide an object that directly exposes
\texttt{LiterateProgramSourceFiles}. This way, we can even use \texttt{scalac} to
work with literate programs.

$\left<\mbox{\emph{*}}\right>\equiv$
\begin{program}{\vem package}~scalit.tangle
\\[0.5em]$<$CoTangle~$-$~send~tangled~to~compiler$>$
\\[0.5em]$<$A~source~file~format~{\vem for}~literate~programs$>$
\\[0.5em]$<$A~{\vem new}~position~{\vem type}$>$
\\[0.5em]$<$LitComp~$-$~the~command~line~application$>$
\\[0.5em]$<$LiterateCompilerSupport~$-$~{\vem object}~{\vem for}~scalac$>$
\\[0.5em]\end{program}


\subsection{CoTangle}
The following class will pass the source files that we get to the compiler (and
maybe a destination directory),
so a first step is to include the compiler class. Later, we'll outline
how to actually compile.

$\left<\mbox{\emph{CoTangle - send tangled to compiler}}\right>\equiv$
\begin{program}{\vem class}~CoTangle$($sourceFiles\,{\rm :}~List$[$LiterateProgramSourceFile$]$,
\\~~~~~~~~~~~~~~~~~~~~~~~~~~destination\,{\rm :}~Option$[$String$]$$)$~{\small\{}
\\~~~~$<$Include~a~compiler$>$
\\~~~~$<$Compile~a~literate~program$>$
\\{\small\}}
\\[0.5em]\end{program}\classdefinition{CoTangle}
\valuedefinition{CoTangle}{sourceFiles}
\valuedefinition{CoTangle}{destination}



The compiler will also have to have a place to report errors to,
so we will need a reporter. The usual way would be to have an object
overriding some behavior for this, but we will just instantiate a
standard reporter.

$\left<\mbox{\emph{Include a compiler}}\right>\equiv$
\begin{program}~~~~{\vem import}~scala.tools.nsc.{\small\{}Global,Settings,reporters{\small\}}
\\~~~~{\vem import}~reporters.ConsoleReporter
\\[0.5em]\end{program}


At the moment, there is only one setting that we might give to the compiler:
If we have a destination directory, then we'll pass it

$\left<\mbox{\emph{Include a compiler}}\right>+\equiv$
\begin{program}~~~~{\vem val}~settings~=~{\vem new}~Settings$($$)$
\\~~~~destination~{\vem match}~{\small\{}
\\~~~~~~~~{\vem case}~Some$($dd$)$~$\Rightarrow$~{\small\{}
\\~~~~~~~~~~~~settings.outdir.tryToSet$($List$($"$-$d",dd$)$$)$
\\~~~~~~~~{\small\}}
\\~~~~~~~~{\vem case}~None~$\Rightarrow$~$($$)$
\\~~~~{\small\}}
\\~~~~{\vem val}~reporter~=
\\~~~~~~~~{\vem new}~ConsoleReporter$($settings,{\vem null},
\\~~~~~~~~~~~~~~~~~~~~~~~~~~~~~~~~~~~~~~~~~~~~~~~~{\vem new}~java.io.PrintWriter$($System.err$)$$)$
\\~~~~{\vem val}~compiler~=~{\vem new}~Global$($settings,reporter$)$
\\[0.5em]\end{program}


\subsection{Literate program as a source file}
After we have instanciated a compiler, we want to feed him a source
file. In this file, we'd like to conserve original line numbering etc.

Much of the functionality will be close to a batch source file. The main
thing that changes is the mapping position $\rightarrow$ position in literate
file.

$\left<\mbox{\emph{A source file format for literate programs}}\right>\equiv$
\begin{program}{\vem import}~scala.tools.nsc.util.{\small\{}BatchSourceFile,Position,LinePosition{\small\}}
\\{\vem class}~LiterateProgramSourceFile$($chunks\,{\rm :}~ChunkCollection$)$
\\~~~~{\vem extends}~BatchSourceFile$($chunks.filename,
\\~~~~~~~~~~~~~~~~~~~~~~~~~~~~~~~~~~~~~~~~~~~~~~~~~~~~chunks.serialize$($"$*$"$)$.toArray$)$~{\small\{}
\\~~~~$<$line~mappings$>$
\\~~~~$<$find~a~line$>$
\\~~~~$<$the~original~source~file$>$
\\~~~~$<$position~in~ultimate~source~file$>$
\\{\small\}}
\\[0.5em]\end{program}\classdefinition{LiterateProgramSourceFile}
\valuedefinition{LiterateProgramSourceFile}{chunks}



\subsection{Position in ultimate source file}
The tangling process rearranges lines from the original literate program
so that they are a valid source input. For compiling purposes, we want to
find out to which original line a line in the tangled file corresponds. As
this will be done quite often, we want to cache the results:

$\left<\mbox{\emph{line mappings}}\right>\equiv$
\begin{program}~~~~{\vem val}~lines2orig~=~{\vem new}~scala.collection.mutable.HashMap$[$Int,Int$]$$($$)$
\\[0.5em]\end{program}


Finding a line amounts to iterating over the stream of code blocks until
we find the one containing the line. This might not be an optimal solution -
now that we are eating the first newline after the chunk definition, it
might actually happen that we point to the wrong line, for example with code
like 

\begin{verbatim}
val s = <Read s>
\end{verbatim}

and

\texttt{$\langle$$\langle$Read s$\rangle$$\rangle$=}
\begin{verbatim}
1/0
\end{verbatim}

Here, the error is in the second part of the code, but the line corresponding
to the first part will be returned. One way to fix this would be to consider
offset information.

$\left<\mbox{\emph{find a line}}\right>\equiv$
\begin{program}~~~~{\vem import}~scalit.markup.StringRefs.\_
\\~~~~{\vem lazy}~{\vem val}~codeblocks\,{\rm :}~Stream$[$RealString$]$~=
\\~~~~~~~~chunks.expandedStream$($"$*$"$)$
\\[0.5em]~~~~{\vem def}~findOrigLine$($ol\,{\rm :}~Int$)${\rm :}~Int~=
\\~~~~~~~~{\vem if}$($~lines2orig~contains~ol~$)$~lines2orig$($ol$)$
\\~~~~~~~~{\vem else}~{\small\{}
\\~~~~~~~~~~~~{\vem def}~find0$($offset\,{\rm :}~Int,
\\~~~~~~~~~~~~~~~~~~~~search\,{\rm :}~Stream$[$RealString$]$$)${\rm :}~Int~=~search~{\vem match}~{\small\{}
\\~~~~~~~~~~~~~~~~{\vem case}~Stream.empty~$\Rightarrow$~error$($"Could~not~find~line~{\vem for}~"~$+$~ol$)$
\\~~~~~~~~~~~~~~~~{\vem case}~Stream.cons$($first,rest$)$~$\Rightarrow$
\\~~~~~~~~~~~~~~~~~~~~first~{\vem match}~{\small\{}
\\~~~~~~~~~~~~~~~~~~~~~~~~{\vem case}~RealString$($cont,from,to$)$~$\Rightarrow$~{\small\{}
\\~~~~~~~~~~~~~~~~~~~~~~~~~~~~{\vem val}~diff~=~to~$-$~from
\\~~~~~~~~~~~~~~~~~~~~~~~~~~~~{\vem if}$($~ol~$\geq$~offset~\&\&~ol~$\leq$~offset~$+$~diff~$)$~{\small\{}
\\~~~~~~~~~~~~~~~~~~~~~~~~~~~~~~~~{\vem val}~res~=~from~$+$~$($ol~$-$~offset$)$
\\~~~~~~~~~~~~~~~~~~~~~~~~~~~~~~~~lines2orig~$+$=~$($ol~$\rightarrow$~res$)$
\\~~~~~~~~~~~~~~~~~~~~~~~~~~~~~~~~res
\\~~~~~~~~~~~~~~~~~~~~~~~~~~~~{\small\}}~{\vem else}
\\~~~~~~~~~~~~~~~~~~~~~~~~~~~~~~~~find0$($offset~$+$~diff,rest$)$
\\~~~~~~~~~~~~~~~~~~~~~~~~{\small\}}
\\~~~~~~~~~~~~~~~~~~~~{\small\}}
\\~~~~~~~~~~~~{\small\}}
\\[0.5em]~~~~~~~~~~~~find0$($0,codeblocks$)$
\\~~~~~~~~{\small\}}
\\[0.5em]\end{program}\methoddefinition{LiterateProgramSourceFile}{findOrigLine}
\valuedefinition{LiterateProgramSourceFile}{codeblocks}



Most of the work was already done in the tangling phase, so we can
just check whether we are inside a string from the source file. For error
reporting, we will want to point to the original source file, but there
is, of course a problem: The source might come from a markup file, or standard
input and not just a literate program. But in any case except reading a literate
program from standard input (which is of dubious utility anyway), we know the
original source file because of the \texttt{@file} directive, which is then given
to us as an argument. This is very suboptimal: We slurp the whole file
for random access:

$\left<\mbox{\emph{the original source file}}\right>\equiv$
\begin{program}~~~~{\vem import}~scala.tools.nsc.util.{\small\{}SourceFile,CharArrayReader{\small\}}
\\~~~~{\vem lazy}~{\vem val}~origSourceFile~=~{\small\{}
\\~~~~~~~~{\vem val}~f~=~{\vem new}~java.io.File$($chunks.filename$)$
\\~~~~~~~~{\vem val}~inf~=~{\vem new}~java.io.BufferedReader$($
\\~~~~~~~~~~~~{\vem new}~java.io.FileReader$($f$)$$)$
\\~~~~~~~~{\vem val}~arr~=~{\vem new}~Array$[$Char$]$$($f.length$($$)$.asInstanceOf$[$Int$]$$)$
\\~~~~~~~~inf.read$($arr,0,f.length$($$)$.asInstanceOf$[$Int$]$$)$
\\~~~~~~~~{\vem new}~BatchSourceFile$($chunks.filename,arr$)$
\\~~~~{\small\}}
\\[0.5em]\end{program}


With all this information, we can finally override the method that tells
us the position in the original source file. To access the original source
file, 

$\left<\mbox{\emph{position in ultimate source file}}\right>\equiv$
\begin{program}~~~~{\vem override}~{\vem def}~positionInUltimateSource$($position\,{\rm :}~Position$)$~=~{\small\{}
\\~~~~~~~~{\vem val}~line~=~position.line~{\vem match}~{\small\{}
\\~~~~~~~~~~~~{\vem case}~None~$\Rightarrow$~0
\\~~~~~~~~~~~~{\vem case}~Some$($l$)$~$\Rightarrow$~l
\\~~~~~~~~{\small\}}
\\~~~~~~~~{\vem val}~col~=~position.column~{\vem match}~{\small\{}
\\~~~~~~~~~~~~{\vem case}~None~$\Rightarrow$~0
\\~~~~~~~~~~~~{\vem case}~Some$($c$)$~$\Rightarrow$~c
\\~~~~~~~~{\small\}}
\\~~~~~~~~{\vem val}~literateLine~=~findOrigLine$($line$)$
\\~~~~~~~~LineColPosition$($origSourceFile,literateLine,col$)$
\\~~~~{\small\}}
\\[0.5em]\end{program}\methoddefinition{LiterateProgramSourceFile}{positionInUltimateSource}



The position class that we return is also defined especially for this
use - we do not want to count the offset into the literate file:

$\left<\mbox{\emph{A new position type}}\right>\equiv$
\begin{program}{\vem import}~scala.tools.nsc.util.SourceFile
\\{\vem case}~{\vem class}~LineColPosition$($source0\,{\rm :}~SourceFile,~line0\,{\rm :}~Int,
\\~~~~~~~~~~~~~~~~~~~~~~~~~~~~~~~~~~~~~~~~column0\,{\rm :}~Int$)$~{\vem extends}~Position~{\small\{}
\\~~~~{\vem override}~{\vem def}~offset~=~None
\\~~~~{\vem override}~{\vem def}~column\,{\rm :}~Option$[$Int$]$~=~Some$($column0$)$
\\~~~~{\vem override}~{\vem def}~line\,{\rm :}~Option$[$Int$]$~~~=~Some$($line0$)$
\\~~~~{\vem override}~{\vem def}~source~=~Some$($source0$)$
\\{\small\}}
\\[0.5em]\end{program}\classdefinition{LineColPosition}
\methoddefinition{LineColPosition}{offset}
\methoddefinition{LineColPosition}{column}
\methoddefinition{LineColPosition}{line}
\methoddefinition{LineColPosition}{source}
\valuedefinition{LineColPosition}{source0 }
\valuedefinition{LineColPosition}{line0 }
\valuedefinition{LineColPosition}{column0 }



\subsection{The compilation process}
With the source file format in place, calling the compiler becomes quite
simple: We create a new \texttt{Run} which will compile the files:

$\left<\mbox{\emph{Compile a literate program}}\right>\equiv$
\begin{program}~~~~{\vem def}~compile\,{\rm :}~Global\#Run~=~{\small\{}
\\~~~~~~~~{\vem val}~r~=~{\vem new}~compiler.Run
\\[0.5em]~~~~~~~~r.compileSources$($sourceFiles$)$
\\~~~~~~~~{\vem if}$($~compiler.globalPhase.name~!=~"terminal"~$)$~{\small\{}
\\~~~~~~~~~~~~System.err.println$($"Compilation~failed"$)$
\\~~~~~~~~~~~~System.exit$($2$)$
\\~~~~~~~~{\small\}}
\\[0.5em]~~~~~~~~r
\\~~~~{\small\}}
\\[0.5em]\end{program}\methoddefinition{CoTangle}{compile}



At the moment, we are not very specific about error reporting: If
compilation does not work, we'll just exit.

\subsection{The command line application}
For this command line application, we just take the list of
chunks from the literate settings. Then fo every such chunk
we'll create a source file. All of these source files are
compiled together and stored in the path given by the \texttt{-d}
command line flag.

$\left<\mbox{\emph{LitComp - the command line application}}\right>\equiv$
\begin{program}{\vem object}~LitComp~{\small\{}
\\~~~~{\vem def}~main$($args\,{\rm :}~Array$[$String$]$$)${\rm :}~Unit~=~{\small\{}
\\~~~~~~~~{\vem import}~scalit.util.LiterateSettings
\\[0.5em]~~~~~~~~{\vem val}~ls~=~{\vem new}~LiterateSettings$($args$)$
\\[0.5em]~~~~~~~~{\vem val}~sourceFiles~=~ls.chunkCollections~map~{\small\{}
\\~~~~~~~~~~~~cc~$\Rightarrow$~{\vem new}~LiterateProgramSourceFile$($cc$)$
\\~~~~~~~~{\small\}}
\\[0.5em]~~~~~~~~{\vem val}~destinationDir\,{\rm :}~Option$[$String$]$~=
\\~~~~~~~~~~~~ls.settings~get~"$-$d"~{\vem match}~{\small\{}
\\~~~~~~~~~~~~~~~~{\vem case}~Some$($x~{\rm :}{\rm :}~xs$)$~$\Rightarrow$~Some$($x$)$
\\~~~~~~~~~~~~~~~~{\vem case}~\_~$\Rightarrow$~None
\\~~~~~~~~~~~~{\small\}}
\\[0.5em]~~~~~~~~{\vem val}~cotangle~=~{\vem new}~CoTangle$($sourceFiles,
\\~~~~~~~~~~~~~~~~~~~~~~~~~~~~~~~~~~~~~~~~~~~~~~~~~~~~~~~~~~~~destinationDir$)$
\\[0.5em]~~~~~~~~cotangle.compile
\\~~~~{\small\}}
\\{\small\}}
\\[0.5em]\end{program}\objectdefinition{LitComp}
\methoddefinition{LitComp}{main}



\subsection{Support for scalac}
When the \texttt{scalac} compiler is invoked, it works directly on \texttt{SourceFile}s. Before these
are parsed, we have to map the literate file to one of concrete code. Fortunately, we already
have a class \texttt{LiterateProgramSourceFile} which serves that purpose. We therefore only provide
one function, taking the filename of the literate source, building the chunks and then returning
a \texttt{LiterateProgramSourceFile}.

$\left<\mbox{\emph{LiterateCompilerSupport - object for scalac}}\right>\equiv$
\begin{program}{\vem object}~LiterateCompilerSupport~{\small\{}
\\~~~~{\vem def}~getLiterateSourceFile$($filename\,{\rm :}~String$)${\rm :}~BatchSourceFile~=~{\small\{}
\\~~~~~~~~{\vem import}~scalit.util.conversions
\\~~~~~~~~{\vem val}~cbs~=~conversions.codeblocks$($conversions.blocksFromLiterateFile$($filename$)$$)$
\\~~~~~~~~{\vem val}~chunks~=~emptyChunkCollection$($filename$)$~addBlocks~cbs
\\~~~~~~~~{\vem new}~LiterateProgramSourceFile$($chunks$)$
\\~~~~{\small\}}
\\{\small\}}
\end{program}\objectdefinition{LiterateCompilerSupport}
\methoddefinition{LiterateCompilerSupport}{getLiterateSourceFile}



\end{document}


