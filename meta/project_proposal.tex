\documentclass[a4paper,12pt]{article}
\usepackage[pdftex,bookmarks=true,bookmarksnumbered=false,
bookmarksopen=false,colorlinks=true,linkcolor=black,anchorcolor=blue]{hyperref}
\title{Literate programming tools for Scala: Project proposal and timeline}
\author{Sebastian Gfeller}
\begin{document}
\maketitle
\section{Introduction}
Literate programming sees programs as a work of literature: Documentation and
source code are closely interspersed, the parts of the source explained in
an order that makes it easy for a human to understand: We want to write
programs for humans to read primarily.

Scala allows for very elegant and succinct programs, but these programs are
still documented in a very traditional style. I think Scala would benefit
from literate programming tools to motivate programmers to write beautiful
code with an explanation in mind.

\section{What I will do}
To write literate programs, we need a way to indicate sections of documentation
and sections of code as well as how to put the code together. I will adapt
the same syntax as noweb because it seems more lightweight than the old
Web syntax.

Two tools are needed, Tangle, which extracts the compilable source from
the literate program and Weave, which extracts the documentation. I will write
these two tools in Scala to avoid too many dependencies. To serve as a first
example, I will directly write it in literate style, first relying on
noweb to extract source and documentation, and later on using the version
that I have written.

After having produced a reasonably stable version of the tools, I will
focus on better integrating them with Scala: Detect where values, classes,
traits and methods are defined and allowing for some pretty printing.

\section{Related work}
\begin{itemize}
\item CWeb ( \href{http://sunburn.stanford.edu/~knuth/cweb.html}{http://sunburn.stanford.edu/~knuth/cweb.html} )

A version of web for C,C++ and Java

\item noweb ( \href{http://www.eecs.harvard.edu/nr/noweb/}{http://www.eecs.harvard.edu/nr/noweb/} )

A language-independent literate programming tool suite. I will use the same
syntax.

\end{itemize}

\section{Timeline}
\begin{description}
\item[2008-03-07] Finished a version of tangle that can extract itself without
relying on noweb anymore.
\item[2008-03-21] Finished a basic version of weave that produces valid
LaTeX output
\item[2008-03-28] Chunk references in documentation, identify class, trait and
object definitions.
\item[2008-04-11] Compiler support. Initial public release
\item[2008-04-25] Refinements: Good Unicode support, pretty printing,
reasonable presentation of class/method index
\item[2008-05-09] Finish final report
\item[2008-05-30] Hand in final report
\end{description}

\section{Deliverables}
As final product, I will produce tangle and weave, ready for inclusion in the
standard Scala distribution. The written part of this semester project will
primarily be the literate version of these programs, accompanied by a
critical evaluation of literate programming techniques in Scala, especially
the applicability of Knuth's original ideas.
\end{document}
